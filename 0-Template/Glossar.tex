%Akronyme

%\newacronym{MOM}{MOM}{Message-oriented Middlware\protect\glsadd{glos:MOM}}

% Glossareinträge

\newglossaryentry{webscale}{
name={Web Scale},
description={Bezeichnet Anwendungen die aufgrund ihrer Architektur mit mehreren Millionen Nutzer zurecht kommen. Das Internet ermöglicht es, global erreichbare Software Systeme bereitzustellen}
}

\newglossaryentry{locationtransparency}{
name={Location Transparency},
description={Ist ein System \enquote{Location Transparent} dürfen die Komponenten und deren Kommunikation nicht von einem physikalischen Host oder anderen \enquote{örtlichen} Gegebenheiten abhängig sein}
}

\newglossaryentry{denialofservice}{
name={Denial of Service Angriff},
description={Ein Denial of Service Angriff hat zum Ziel eine Anwendung beispielsweise durch eine überdurchschnittliche Anzahl von Anfragen mutwillig unnutzbar zu machen}
}

\newglossaryentry{messagequeue}{
name={Message Queue},
description={Eine Message Queue ist eine Datenstruktur bzw. ein Zwischenspeicher für Nachrichten in einer bestimmten Reihenfolge. Diese werden bis zu deren Abarbeitung in der Queue vorgehalten}
}

\newglossaryentry{flowcontrol}{
name={Flow Control},
description={Als Flow Control werden allgemein unterschiedlichste Verfahren bezeichnet, mit denen der \enquote{Fluss} von Daten innerhalb eines System gesteuert und beeinflusst werden kann}
}

\newglossaryentry{concurrency}{
name={Concurrency},
description={Concurrency oder zu Deutsch Nebenläufigkeit ist die Fähigkeit eines Systems mehrere Operationen gleichzeitig oder scheinbar gleichzeitig ausführen zu können. Dabei können diese Operationen auch im Zusammenhang stehen}
}

\newglossaryentry{pipelining}{
name={Pipelining},
description={Beim Pipelining zerlegt ein Prozessor ein Maschinenbefehl in Teilaufgaben, die dann zum Teil parallel durchgeführt werden können}
}

\newglossaryentry{multitasking}{
name={Multitasking},
description={Multitasking bezeichnet die Fähigkeit eines Betriebsystems, mehrere Prozess quasi-parallel auszuführen. Die Prozesses werden in in sehr kurzen Zeitabständen abwechselnd Rechnenleistung zugewiesen. Es entsteht der Eindruck als würden die Prozesse gleichzeitig ausgeführt}
}