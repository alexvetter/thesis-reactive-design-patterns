% Glossareinträge

\newglossaryentry{webscale}{
name={Web Scale},
description={Der Begriff Web Scale bezeichnet Anwendungen, die aufgrund ihrer Architektur mit mehreren Millionen Nutzer zurechtkommen. Das Internet ermöglicht es, global erreichbare Software Systeme bereitzustellen}
}

\newglossaryentry{locationtransparency}{
name={Location Transparency},
description={Ist ein System \enquote{Location Transparent}, dürfen die Komponenten und deren Kommunikation nicht von einem physikalischen Host oder anderen \enquote{örtlichen} Gegebenheiten abhängig sein}
}

\newglossaryentry{denialofservice}{
name={Denial of Service Angriff},
description={Ein Denial of Service Angriff hat zum Ziel eine Anwendung beispielsweise durch eine überdurchschnittliche Anzahl von Anfragen mutwillig unnutzbar zu machen}
}

\newglossaryentry{messagequeue}{
name={Message Queue},
description={Eine Message Queue ist eine Datenstruktur bzw. ein Zwischenspeicher für Nachrichten in einer bestimmten Reihenfolge. Diese werden bis zu deren Abarbeitung in der Queue vorgehalten}
}

\newglossaryentry{flowcontrol}{
name={Flow Control},
description={Als Flow Control werden allgemein unterschiedlichste Verfahren bezeichnet, mit denen der \enquote{Fluss} von Daten innerhalb eines Systems gesteuert und beeinflusst werden kann}
}

\newglossaryentry{concurrency}{
name={Concurrency},
description={Concurrency oder zu Deutsch Nebenläufigkeit ist die Fähigkeit eines Systems mehrere Operationen gleichzeitig oder scheinbar gleichzeitig ausführen zu können. Dabei können diese Operationen auch im Zusammenhang stehen}
}

\newglossaryentry{pipelining}{
name={Pipelining},
description={Beim Pipelining zerlegt ein Prozessor ein Maschinenbefehl in Teilaufgaben, die dann zum Teil parallel durchgeführt werden können}
}

\newglossaryentry{multitasking}{
name={Multitasking},
description={Multitasking bezeichnet die Fähigkeit eines Betriebssystems, mehrere Prozesse quasi-parallel auszuführen. Den Prozessen wird in sehr kurzen Zeitabständen abwechselnd Rechenleistung zugewiesen. Es entsteht der Eindruck, als würden die Prozesse gleichzeitig ausgeführt}
}

\newglossaryentry{mutualexclusion}{
name={Mutal Exclusion},
description={Mutal Exclusion oder wechselseitiger Ausschluss beschreibt verschiedene Verfahren, mit denen sichergestellt wird, dass kritische Abschnitte in einem Programm gleichzeitig oder verschränkt durch nebenläufige Prozesse ausgeführt werden}
}

\newglossaryentry{memorycorruption}{
name={Memory Corruption},
description={Memory Corruption entsteht durch ungewollte Veränderung von Datenstrukturen aufgrund von Fehlern in einem Programm}
}

\newglossaryentry{deadlock}{
name={Deadlock},
description={Der Begriff Deadlock bezeichnet einen ungewollten Zustand, bei dem eine zyklische Wartesituation zwischen mehreren Prozessen oder Threads auftritt. Jeder involvierte Prozess oder Thread wartet auf die Zuweisung einer Ressource, die jedoch durch einen anderen Teilnehmer belegt ist}
}

\newglossaryentry{starvation}{
name={Starvation},
description={Starvation beschreibt ein Problem der Concurrency, bei dem ein Prozess oder ein Thread benötigte Ressourcen nicht zugewiesen bekommt. Folglich kann dieser seine Arbeit nicht verrichten}
}

\newglossaryentry{contextswitching}{
name={Context Switching},
description={Context Switching beschreibt den Vorgang eines Betriebssystems, bei dem ein Task, also ein Prozess oder ein Thread, unterbrochen wird und Rechenkapazität einem anderen Task zugeordnet wird. Dabei wird der Zustand und Ablauf des Tasks gesichert und durch den bereits gesicherten Context des anderen Tasks ersetzt}
}

\newglossaryentry{loadbalancing}{
name={Load Balancing},
description={Mittels Load Balancing werden Anfragen auf mehrere redundante Systeme verteilt. Dies kann auf Hardware- oder Softwareebene erfolgen. Die Lastverteilung ist notwendig, da einzelne Server oder Prozesse nur eine gewisse Anzahl von Anfragen verarbeiten können. Oft wird Load Balancing auch genutzt, um Ausfallsicherheit zu gewährleisten}
}

\newglossaryentry{connectionpool}{
name={Connection-Pool},
description={Ein Connection Pool ist ein Cache bzw. ein Zwischenspeicher aus bestehenden Verbindungen beispielsweise zu einer Datenbank. Das Aufbauen einer Verbindung ist aufwendig. Anstatt bei jeder Anfrage eine neue Verbindung aufzubauen, können Verbindungen aus dem Pool entnommen werden und nach Fertigstellung der Anfrage wieder zurück- bzw. freigegeben werden}
}