%HEADER

% KOMA-Script-Klasse: scrreprt
% deutsches Design, Schriftgröße 12, DIN A4
% Literaturverzeichnis und Index in Inhaltsverzerzeichnis einbinden
% Einseitig
\documentclass[12pt,a4paper,listof=totoc,oneside]{scrreprt}

% Seitenspiegel einstellen
\usepackage[a4paper]{geometry}
\geometry{a4paper,left=30mm,right=25mm,bottom=20mm,top=15mm,bindingoffset=2mm,includehead,includefoot}

\def\makroFH-Kopfzeilenstil{
	\pagestyle{scrheadings}
	\setheadsepline{0.4pt}
	\pagestyle{scrheadings}
	\renewcommand*{\chapterpagestyle}{scrheadings}
}

% schöneres titlespacing
\usepackage{titlesec}
\titlespacing*{\subsubsection}{0pt}{1.3ex}{0pt}

% passende Codierung
\usepackage[utf8]{inputenc}

% schalte Umlaute frei
\usepackage[ngerman]{babel}

% wird gebraucht für richtige Umlaute 1)
\usepackage{pslatex}

% wird gebraucht für richtige Umlaute 1)
\usepackage[T1]{fontenc}

% Mathematik
\usepackage{amsmath}

% Symbole
\usepackage{amssymb}

% Griechische Symbole
\usepackage{upgreek}

% weitere Symbole
\usepackage{pxfonts}

% Phonetischen Alphabete für LaTeX
\usepackage{tipa}

% farbige Schriften
\usepackage{color}

% scrhack redefines macros of packages from other authors!
\usepackage{scrhack}

% Bilder fixieren
\usepackage{float}

% Grafiken einbinden
\usepackage{graphicx}

% Kopf- und Fußzeilen
\usepackage[automark,standardstyle,markusedcase]{scrpage2}

% deutsche Überschriften
\usepackage[ngerman]{translator}

% deutsche Anführungszeichen
\usepackage[autostyle=true,german=quotes]{csquotes}

% Kopfzeilenabstand festlegen
\setlength{\headheight}{10mm}

% Captions
\usepackage{caption3}
\usepackage{caption}

% Für Deutsch Abb. & Tab.
\addto\captionsngerman{
	\renewcommand{\figurename}{Abb.}
	\renewcommand{\tablename}{Tab.}
}

% Zitieren
\usepackage[backend=biber,style=alphabetic,natbib=true]{biblatex}
\bibliography{0-Template/Bibliothek.bib}
\nocite{*}

% für URLs
\usepackage{url}

% Links sollen nicht farbig sein
\usepackage[colorlinks=false,unicode]{hyperref}

% Glossar-Package (keine Seitenzahlen, Abkürzungsverzeichnis, ins Inhaltsverzeichnis)
\usepackage[utf8]{inputenc}
\usepackage[xindy={language=german,codepage=duden-utf8},nonumberlist,acronym,toc]{glossaries}
\makeglossaries

% Code Beispiele
\usepackage{listings}
\renewcommand{\lstlistlistingname}{Quellcodeverzeichnis}
\renewcommand{\lstlistingname}{Quellcode}

% Style für Listings
\lstset{frame=single,
captionpos=t,
numbers=left,
breaklines=true,
prebreak=\raisebox{0ex}[0ex][0ex]{\ensuremath{\hookleftarrow}},
numberstyle=\tiny,
basicstyle=\fontfamily{pcr}\selectfont\footnotesize}

% Um besser Tabellen zu gestalten
\usepackage{colortbl}
\definecolor{light-gray}{gray}{.8}
\definecolor{lighter-gray}{gray}{.9}
\definecolor{lightgreen}{rgb}{0.56, 0.93, 0.56}

\usepackage{booktabs}
\usepackage{tabularx}

% Dokument interne Referenzen
\newcommand*{\fullref}[1]{\hyperref[{#1}]{\nameref*{#1}~\ref{#1}}}

% Fügt bei Zahlen kleiner 10  eine 0 vorne dran
\newcommand{\leadingzero}[1]{\ifnum #1<10 0\the#1\else\the#1\fi}

% ISO Datum im Format YYYY-MM-DD
\newcommand{\todayiso}{\the\year–\leadingzero{\month}–\leadingzero{\day}}

% Kein Einrücken beim Beginn eines Paragraphen (z.B. nach einem Bild)
\setlength{\parindent}{0pt}

% Rahmen um Text
\usepackage{framed}

% Inkludiere Einstellungen
\newcommand{\thesis}{Masterarbeit}
\newcommand{\name}{Vetter}
\newcommand{\vorname}{Alexander}

\newcommand{\batitle}{Reactive Design Patterns}
\newcommand{\babetreuer}{Prof.\ Dr.\ Gudrun Schiedermeier}

