% !TEX root = ../Masterdatei.tex
\begin{abstract}
\begin{center}
\Huge
\textit{\textbf{Zusammenfassung}}
\end{center}
\normalsize
\vspace{15mm}

\begin{itshape}
Die vorliegende Arbeit beschäftigt sich mit Design Patterns, die die Eigenschaften des Reactive Manifestos erfüllen. Ziel der Arbeit wird es somit sein, Design Patterns zu identifizieren, die bei der Entwicklung von reaktiven Systemen von Nutzen sein können.\\
Zu Beginn werden neu entstandene Anforderungen an Software Systeme erläutert, die die Motivation hinter dem Reactive Manifesto erklären.\\
Für den Hauptteil werden die Grundlagen zu funktionaler Programmierung sowie den genauen Eigenschaften reaktiver Systeme dargelegt.\\
Im Hauptteil werden dann eine Auswahl an Design Patterns genauer beleuchtet und festgestellt, ob diese die Eigenschaften des Reactive Manifestos erfüllen. Dazu werden dann Code Beispiele folgen, die den Umgang mit den Patterns in der jeweiligen Sprache darstellen sollen.\\
Zum Schluss werden die Ergebnisse zusammen gefasst und ein Ausblick auf die Verwendung von reaktiven Systemen in moderner Softwarearchitektur gegeben.
\end{itshape}

\end{abstract}
