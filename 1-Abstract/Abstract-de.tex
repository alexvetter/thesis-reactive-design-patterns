% !TEX root = ../Masterarbeit.tex
\begin{abstract}
\begin{center}
\Huge
\textit{\textbf{Abstract}}
\end{center}
\normalsize
\vspace{15mm}

Die vorliegende Arbeit beschäftigt sich mit Design Patterns, die die Eigenschaften des Reactive Manifestos erfüllen. Ziel der Arbeit ist es, Design Patterns zu identifizieren, die bei der Entwicklung von reaktiven Systemen von Nutzen sind.
Zu Beginn werden neu entstandene Anforderungen an Software Systeme erläutert, die die Motivation hinter dem Reactive Manifesto und der Arbeit erklären.
Darauffolgend werden Grundlagen bezüglich der Eigenschaften reaktiver Systeme, Parallelität \& Concurrency sowie funktionaler Programmierung geschaffen.
Im Hauptteil folgt dann eine Auswahl an Design Patterns, die genauer beleuchtet und im Bezug auf das Reactive Manifesto klassifiziert werden. Überdies werden zwei Concurrency Modelle diskutiert, welche vor allem für reaktive Systeme von Vorteil sind.
Zum Schluss folgt eine persönliche Einschätzung in welchen Bereichen reaktive Systeme sinnvoll sind sowie eine Zusammenfassung der Arbeit.

\end{abstract}

% Thema und Zielsetzung
% Reaktive Design Pattern; Auswahl an Patterns und welche positive Eigenschaft haben sie, um die Entwicklung reaktiver Systeme zu unterstützen.

% Theorie
% Die Arbeit wird beantworten, ob und welche Eigenschaften des Reaktive Manifestos von einem Design Pattern erfüllt werden.

% Fragestellung
% Die Arbeit wird sich mit Reaktive Design Patterns beschäftigen, also ob und welche es gibt.

% Ergebnis & Fazit
% Fazit wird sein, dass es wichtig ist sich mit Reaktiven Systemen auseinander zu setzen, da sich die Anforderungen an Software bereits im Wandlen befinden. Concurrency und Parallelität gewinnen immer mehr an Bedeutung.