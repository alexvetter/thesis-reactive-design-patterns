% !TEX root = ../Masterdatei.tex
\begin{abstract}
\begin{center}
\Huge
\textit{\textbf{Abstract}}
\end{center}
\normalsize
\vspace{15mm}

Die vorliegende Arbeit beschäftigt sich mit Design Patterns, die die Eigenschaften des Reactive Manifestos erfüllen. Ziel der Arbeit wird es sein, Design Patterns zu identifizieren, die bei der Entwicklung von reaktiven Systemen von Nutzen sein können. Zu Beginn werden neu entstandene Anforderungen an Software Systeme erläutert, die die Motivation hinter dem Reactive Manifestos erklären. Für den Hauptteil werden die Grundlagen zu funktionaler Programmierung sowie den genauen Eigenschaften reaktiver Systeme dargelegt. Im Hauptteil werden dann eine Auswahl an Design Patterns genauer beleuchtet und festgestellt, ob diese die Eigenschaften des Reactive Manifestos erfüllen. Dazu werden dann Code Beispiele folgen, die den Umgang mit den Patterns in der jeweiligen Sprache darstellen sollen. Zum Schluss werden die Ergebnisse zusammen gefasst und ein Ausblick auf die Verwendung von reaktiven Systemen in moderner Softwarearchitektur gegeben.

\end{abstract}

% Thema und Zielsetzung
% Reaktive Design Pattern; Auswahl an Patterns und welche positive Eigenschaft haben sie, um die Entwicklung reaktiver Systeme zu unterstützen.

% Theorie
% Die Arbeit wird beantworten, ob und welche Eigenschaften des Reaktive Manifestos von einem Design Pattern erfüllt werden.

% Fragestellung
% Die Arbeit wird sich mit Reaktive Design Patterns beschäftigen, also ob und welche es gibt.

% Ergebnis & Fazit
% Fazit wird sein, dass es wichtig ist sich mit Reaktiven Systemen auseinander zu setzen, da sich die Anforderungen an Software bereits im Wandlen befinden. Concurrency und Parallelität gewinnen immer mehr an Bedeutung.