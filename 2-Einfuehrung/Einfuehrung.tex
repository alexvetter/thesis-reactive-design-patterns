% !TEX root = ../Masterarbeit.tex
\chapter{Einführung}\label{sec:einfuehrung}
Software wird entwickelt, um Probleme zu lösen und Aufgaben automatisiert zu erledigen --- ohne dabei menschliche Fehler zu machen. Deshalb ist es nicht verwunderlich, dass man an Software die Anforderung stellt, in angemessener Zeit auf Eingaben zu reagieren. Hierfür gibt es den Begriff \textit{responsiveness} (dt.\ Antwortbereitschaft)~\cite[S.~18]{kuhn_reactive_2015}.\\
Ein System, welches auf externe Ereignisse bzw. Eingaben antwortet (reagiert), nennt man reaktives System. David Harel und Amir Pnueli unterscheiden in ihrer Arbeit \enquote{On the Development of Reactive Systems} (1985) zwischen transformativen und reaktiven Systemen~\cite{harel_development_1985}. Transformative Systeme berechnen Ergebnisse einmalig auf der Basis bestimmer Eingabewerte (z.B. Compiler)~\cite[S.~2]{carkci_dataflow_2014}~\cite{wieringa_design_2003}. Reaktive Systeme definieren sie hingegen wie folgt:

\begin{quotation}
  Reactive systems [...] are repeatedly prompted by the outside world and their role is to continuously respond to external inputs~\cite{harel_development_1985}.
\end{quotation}

Folglich muss ein reaktives System, für die Dauer der Ausführung auf jede Eingabe bzw. Anfrage zeitnah reagieren. Üblicherweise handelt es sich bei reaktiven Systemen, um länger oder durchgehend laufende Applikationen. Hierzu zählen vor allem serverbasierte Anwendungen, die viele und häufig auch parallele Anfragen verarbeiten müssen.\\
In dieser Arbeit wird die Entwicklung von reaktiven Software Systeme behandelt, bei welchen die erwähnte \textit{responsiveness} und weitere Anforderungen von großer Bedeutung sind.

\pagebreak

\section{Wandel der Anforderungen}
Die heutigen Anforderungen an Software unterliegen dem rapiden Wachstum der Nutzung von Software im technischen, wie auch im sozialen Sinne. Der Wandel hin zu einer digitalen Gesellschaft und der weltweiten Vernetzung durch das Internet machen Software allgegenwärtig. Die Art und Weise, wie moderne Systeme implementiert werden, unterliegt ebenfalls diesem Wandel. Um den heutigen Ansprüchen zu genügen, müssen diese robuster, skalierbarer und anpassungsfähiger sein, als es früher der Fall war~\cite{boner_reactive_2014}.\\

Früher wurden mithilfe von Software überwiegend mathematische Probleme gelöst. Heutzutage sind die Aufgaben viel komplexer, nicht im Bezug auf mathematische Probleme, sondern vielmehr im Bezug auf die Menge der Daten~\cite[S.~18]{kuhn_reactive_2015}. Zum Beispiel muss der Kurznachrichten Dienst Twitter täglich über 500 Millionen Tweets\footnotemark[1] verarbeiten und über 300 Millionen Nutzern\footnotemark[2] auf der ganzen Welt zur Verfügung stellen. Netflix war im Dezember 2015 für rund 35~\%\footnotemark[3] des gesamten Internet-Traffics in Nordamerika verantwortlich. Das Internet ermöglicht es, global erreichbare Software Systeme bereitzustellen. Dies führt dazu, dass diese Systeme theoretisch mit ca. 3 Milliarden Nutzern\footnotemark[4] zurechtkommen müssen. Aufgrund der Zeitzonen existieren keine Wartungszeiträume, wie bei üblichen Software Installationen.

\footnotetext[1]{Twitter. Anzahl der täglichen Tweets auf Twitter vom Februar 2010 bis Oktober 2013 (in Millionen). http://de.statista.com/statistik/daten/studie/237226 (zugegriffen am 09. Januar 2016).}
\footnotetext[2]{Twitter. Anzahl der monatlich aktiven Nutzer von Twitter weltweit vom 1. Quartal 2010 bis zum 3. Quartal 2015 (in Millionen). http://de.statista.com/statistik/daten/studie/232401 (zugegriffen am 09. Januar 2016).}
\footnotetext[3]{Sandvine. Anwendungen mit dem größten Anteil am Internet-Datenverkehr in Nordamerika im Dezember 2015. http://de.statista.com/statistik/daten/studie/294315 (zugegriffen am 3. Juni 2016).}
\footnotetext[4]{Internet Live Stats. Anzahl der Internetnutzer weltweit von 1997 bis 2014 (in Millionen). http://de.statista.com/statistik/daten/studie/186370 (zugegriffen am 16. Februar 2016).}

\pagebreak

Die neuen Anforderungen, die durch das Internet und durch die rasant steigende Anzahl der Nutzer entstehen fast man unter dem Begriff \enquote{\gls{webscale}} zusammen. Die dafür nötigen Architekturansätze solcher skalierbarer Systeme halten auch Einzug in die Entwicklung von Enterprise Applikationen. Unternehmen benötigen mehr und mehr die technischen Konzepte, wie sie beispielsweise von Twitter und Netflix umgesetzt werden. Denn viele Unternehmen müssen mittlerweile mit immer größer werdenden Datenmengen zurechtkommen.\\

Neben der Skalierbarkeit unterliegen auch die Antwortzeiten dem Wandel der Anforderungen. Der Benutzer von heute ist es nicht mehr gewohnt, lange auf Ergebnisse warten zu müssen. Laut einer Umfrage unter Smartphone-Nutzern von 2011 würden 16~\% der Befragten eine aufgerufene Webseite wieder verlassen, wenn die Ladezeit mehr als drei Sekunden beträgt. Weitere 10~\% bei einer Ladezeit von mehr als vier Sekunden und nochmal weitere 28~\% bei mehr als fünf Sekunden\footnotemark[5]. Man kann davon ausgehen, dass Nutzer heute weitaus kürzere Ladezeiten erwarten und fordern.\\

\footnotetext[5]{Wie lange sind Sie bereit zu warten bis eine mobile Website auf Ihrem Smartphone vollständig geladen hat?. http://de.statista.com/statistik/daten/studie/202650 (zugegriffen am 09. Januar 2016).}

Ein weiterer Umbruch ist bei der Entwicklung von Prozessoren zu beobachten. Die Prozessorhersteller sind bei den Frequenzen bzw. Taktraten an Grenzen gestoßen, weshalb man seit einigen Jahren auf Multicore-Architekturen setzt. Für Software Entwickler bedeutet das, um Anwendungen schneller zu machen, müssen diese auf die Multicore-Prozessoren hin optimiert werden~\cite[S. 15]{butcher_seven_2014}. Anwendungen müssen ihre Aufgaben und Teilaufgaben nebenläufig und parallel ausführen, um die Prozessoren optimal auszulasten. Man ist auf einmal mit ähnlichen Problemen konfrontiert, wie bei verteilten Systemen, da man auch bei lokaler Interprozesskommunikation mit Latenzen rechnen muss.

\pagebreak

\section{Reactive Manifesto}
Um den erwähnten Anforderungen gerecht zu werden, muss sich die moderne Software Architektur anpassen. Bonér et al. postulieren in dem Reactive Manifesto, reaktive Systeme wären skalierbarer, einfacher weiterzuentwickeln und zuverlässiger~\cite{boner_reactive_2014}. Aufgrund der im Manifest definierten Eigenschaften bzw. Anforderungen sollen reaktive Systeme den heutigen Herausforderungen besser gewachsen sein. Entspricht die Architektur einer Software dem Reactive Manifesto, gewährleistet man ein System welches \textit{resilient} (dt.~widerstandsfähig), sowie \textit{elastic} (dt.~elastisch) ist und deshalb auch immer \textit{responsive} (dt.~antwortbereit). Um das zu erreichen, muss ein System \textit{message-driven} (dt.~nachrichtenorientiert) sein~\cite[S.~5]{vernon_reactive_2016}. Man bezeichnet diese Anwendungen dann als reaktive Anwendungen im Sinne des Manifests~\cite{boner_reactive_2014}.\\
Das Reactive Manifesto beschreibt keine neuen Konzepte oder Paradigmen, vielmehr formalisiert es Begriffe und bietet technologieübergreifende Definitionen für reaktive Systeme.\\

Die Arbeit soll die Frage beantworten, welche Konzepte und Design Patterns helfen können, reaktive Systeme zu entwickeln.
