% !TEX root = ../Masterarbeit.tex
\chapter{Einführung}\label{sec:einfuehrung}
Software wird entwickelt, um Probleme zu lösen und Aufgaben automatisiert zu erledigen --- ohne dabei menschliche Fehler zu machen. Deshalb ist es nicht verwunderlich, dass man an Software die Anforderung stellt, in angemessener Zeit auf Eingaben zu reagieren. Hierfür gibt es den Begriff \textit{responsiveness} (dt.\ Antwortbereitschaft)~\cite[S.~18]{kuhn_reactive_2015}.\\
Ein System welches auf externe Ereignisse bzw. Eingaben antwortet (reagiert), nennt man reaktives System. David Harel und Amir Pnueli unterschieden in ihrer Arbeit \enquote{On the Development of Reactive Systems} (1985) zwischen reaktiven und transformativen Systemen~\cite{harel_development_1985}. Sie definierten reaktive Systeme wie folgt:

\begin{quotation}
  Reactive systems [...] are repeatedly prompted by the outside world and their role is to continuously respond to external inputs~\cite{harel_development_1985}.
\end{quotation}

Transformative Systeme hingegen berechnen Ergebnisse einmalig auf Basis bestimmter Eingabewerte (z.B. Compiler)~\cite[S.~2]{carkci_dataflow_2014}~\cite{wieringa_design_2003}. Bei reaktiven Systemen handelt es sich üblicherweise um durchgehend laufende Anwendungen, wie zum Beispiel eine Web Applikation.\\
In dieser Arbeit wird die Entwicklung von reaktiven Software Systeme behandelt, bei welchen die erwähnte \textit{responsiveness} und weitere Anforderungen von großer Bedeutung sind.

\pagebreak

\section{Wandel der Anforderungen}
Die heutigen Anforderungen an Software unterliegen dem rapiden Wachstum der Nutzung von Software im technischen wie auch im sozialen Sinne. Der Wandel hin zu einer digitalen Gesellschaft und der weltweiten Vernetzung durch das Internet machen Software allgegenwertig. Die Art und Weise wie moderne Systeme implementiert werden, unterliegt ebenfalls diesem Wandel. Um den heutigen Ansprüchen zu genügen, müssen diese robuster, skalierbarer und anpassungsfähiger sein, als es früher der Fall war~\cite{boner_reactive_2014}.\\

Früher wurden mithilfe von Software überwiegend mathematische Probleme gelöst. Heutzutage sind die Aufgaben viel komplexer, nicht im Bezug auf mathematische Probleme sondern vielmehr im Bezug auf der Menge der Daten~\cite[S.~18]{kuhn_reactive_2015}. Zum Beispiel muss der Kurznachrichten Dienst Twitter täglich über 500 Millionen Tweets\footnotemark[1] verarbeiten und über 300 Millionen Nutzern\footnotemark[2] auf der ganzen Welt zur Verfügung stellen. Man spricht hier von dem sogenannten \textit{Web-scale}. Das Internet ermöglicht es global erreichbare Software System bereitzustellen. Dies führt dazu, dass diese Systeme theoretisch mit ca. 3 Milliarden Nutzern\footnotemark[3] zurechtkommen müssen und aufgrund der Zeitzonen keine Wartungszeiträume existieren. Die Architekturansätze solch skalierbarer Systeme halten auch Einzug in Enterprise Applikationen, da diese mit immer größer werdenden Datenmengen zurechtkommen müssen.

\footnotetext[1]{Twitter. Anzahl der täglichen Tweets auf Twitter vom Februar 2010 bis Oktober 2013 (in Millionen). http://de.statista.com/statistik/daten/studie/237226/umfrage/wachstum-von-twitter-nach-anzahl-der-taeglichen-tweets/ (zugegriffen am 09. Januar 2016).}
\footnotetext[2]{Twitter. Anzahl der monatlich aktiven Nutzer von Twitter weltweit vom 1. Quartal 2010 bis zum 3. Quartal 2015 (in Millionen). http://de.statista.com/statistik/daten/studie/232401/umfrage/monatlich-aktive-nutzer-von-twitter-weltweit-zeitreihe/ (zugegriffen am 09. Januar 2016).}
\footnotetext[3]{Internet Live Stats. Anzahl der Internetnutzer weltweit von 1997 bis 2014 (in Millionen). http://de.statista.com/statistik/daten/studie/186370/umfrage/anzahl-der-internetnutzer-weltweit-zeitreihe/ (zugegriffen am 16. Februar 2016).}

\pagebreak

Neben der Skalierbarkeit unterliegt auch die Antwortzeit dem Wandel der Anforderungen. Der Benutzer von heute ist es nicht mehr gewohnt lange auf Ergebnisse zu warten. Laut einer Studie von 2011 würden 16~\% der Befragten eine aufgerufene Webseite wieder verlassen, wenn die Ladezeit mehr als drei Sekunden beträgt. Weitere 10~\% bei einer Ladezeit mehr als vier Sekunden\footnotemark[4]. Man kann davon ausgehen, dass Nutzer heute weitaus kürzere Ladezeit erwarten und fordern.\\

\footnotetext[4]{Wie lange sind Sie bereit zu warten bis eine mobile Website auf Ihrem Smartphone vollständig geladen hat?. http://de.statista.com/statistik/daten/studie/202650/umfrage/wartezeit-bis-zum-verlassen-einer-mobilen-website/ (zugegriffen am 09. Januar 2016).}

Eine weitere Entwicklung ist bei den Prozessoren zu beobachten. Die Prozessorhersteller sind bei den Frequenzen bzw. Taktraten an Grenzen gestoßen, weshalb man seit einigen Jahren auf Multicore-Architekturen setzt. Für Software Entwickler bedeutet das, um Anwendungen schneller zu machen, müssen diese auf die Multicore-Prozessoren hin optimiert werden~\cite[S. 15]{butcher_seven_2014}. Anwendungen müssen ihre Aufgaben und Teilaufgaben nebenläufig und parallel ausführen, um die Prozessoren optimal auszulasten. Man ist auf einmal mit ähnlichen Problemen konfrontiert wie bei verteilten Systemen, da man auch bei lokaler Interprozesskommunikation mit Latenzen rechnen muss.

\pagebreak

\section{Reactive Manifesto}
Um den erwähnten Anforderungen gerecht zu werden, muss sich die moderne Software Architektur anpassen. Bonér et al. postulieren in dem \enquote{Reactive Manifesto}, reaktive Systeme währen skalierbarer, einfacher weiterzuentwickeln und zuverlässiger~\cite{boner_reactive_2014}. Aufgrund der im Manifest definierten Eigenschaften bzw. Anforderungen sollen reaktive Systeme den heutigen Herausforderungen besser gewachsen sein. Entspricht die Architektur einer Software dem Reactive Manifesto, gewährleistet man ein System welches \textit{resilient} (dt.~widerstandsfähig), sowie \textit{elastic} (dt.~elastisch) ist und deshalb auch immer \textit{responsive} (dt.~antwortbereit). Um das zu erreichen, muss ein System \textit{message-driven} (dt.~nachrichtenorientiert) sein~\cite[S.~5]{vernon_reactive_2016}. Man bezeichnet diese Anwendungen dann als reaktive Anwendungen im Sinne des Manifests~\cite{boner_reactive_2014}.\\
Das Reactive Manifesto beschreibt keine neuen Konzepte oder Paradigmen, vielmehr formalisiert es Begriffe und bietet technologieübergreifende Definitionen für reaktive Systeme.\\
Die Arbeit soll die Frage beantworten, welche Konzepte und Design Patterns helfen reaktive Systeme zu entwickeln.
