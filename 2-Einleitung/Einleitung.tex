% !TEX root = ../Masterdatei.tex
\chapter{Einleitung}\label{sec:einleitung}

Seit dem Software entwickelt wird, dient diese einem gewissen Zweck. Software löst Probleme und erledigt Aufgaben automatisiert und ohne menschliche Fehler zu machen. Eine offensichtliche aber trotzdem sehr wichtige Eigenschaft einer Software ist, dass diese auf jegliche Eingaben in angemessener Zeit reagiert. Hierfür gibt es den Begriff Antwortbereitschaft (engl.\ responsivness). Diese Eigenschaft steht im Kern der Reaktiven Programmierung (engl.\ reactive programming). Der Begriff der Reaktiven Programmierung geht zurück auf das Paper \enquote{On the Development of Reactive Systems} von David Harel und Amir Pnueli aus dem Jahr 1985.~\cite{carkci_dataflow_2014}

\begin{quotation}
  Reactive systems [...] are repeatedly prompted by the outside world and their role is to continuously respond to external inputs.~\cite{harel_development_1985}
\end{quotation}

Die Antwortbereitschaft ist nach wie vor von Bedeutung. Deshalb steht sie im Kern der Reaktiven Programmiung (engl.\ reactive programming), die zu dem weitere Anforderungen definiert.~\cite[S. 18]{kuhn_reactive_2015}

\section{Wandel der Anforderungen}
Die heutigen Anforderungen an Software unterliegen dem rapiden Wachstum der Nutzung von Software im technischen wie auch im sozialen Sinne. Der Wandel hin zu einer digitalen Gesellschaft und der weltweiten Vernetzung durch das Internet machen Software allgegenwertig. Die Art und Weise wie moderne Systeme implementiert werden, unterliegt ebenfalls diesem Wandel. Um den heutigen Ansprüchen zu genügen, müssen diese robuster, skalierbarer und anpassungsfähiger sein, als es früher der Fall war.~\cite{boner_reactive_2014}

Früher hatte man überwiegend mathematische Probleme die gelöst werden mussten. Heutzutage sind die Aufgaben viel komplexer, nicht im Bezug auf mathematische Probleme sondern vielmehr im Bezug auf die Menge der Daten. Zum Beispiel muss der Kurznachrichten Dienst Twitter täglich über 400 Millionen Tweets verarbeiten und Nutzern auf der ganzen Welt zur Verfügung stellen. Hier wird ein weiteres Problem deutlich: \enquote{Nutzern auf der ganzen Welt}. Das Internet und deren Dienste werden von Millionen Nutzern aus der ganzen Welt genutzt. Verteilte Systeme zu konzipieren, ist keine neue Herausforderung, jedoch werden mittlerweile ganz andere Anforderungen an sie gestellt.\\
Eine weitere Entwicklung ist bei den Prozessoren zu beobachten. Prozessoren sind bei den Frequenzen an physikalische Grenzen gestoßen, weshalb man seit einigen Jahren auf Multi-core Prozessoren setzt. Es ist nur sinnvoll, dass Anwendungen auf die Multi-core Prozessoren hin optimiert werden. Ruft man sich nun die anfänglich erwähnte Anforderung \textit{responsivness} wieder ins Gedächtnis, ist die Umsetzung in einem globalen Netz eine durch aus komplizierte Aufgaben.~\cite[S. 15]{butcher_seven_2014}\\
Die Autoren des Reaktiven Manifests postuliert reaktive Systeme wären skalierbarer, einfacher weiterzuentwickeln und zuverlässiger.~\cite{boner_reactive_2014}

\section{Reactive Manifesto}

Typischerweise werden größere Applikationen mit dem \enquote{Single-threaded} Gedanken entworfen und umgesetzt. Das Reactive Manifesto definiert welche Eigenschaften eine Applikationen haben sollte, um reaktiv zu sein. Entspricht die Architektur einer Software dem Reactive Manifesto, gewährleistet man ein System welches antwortbereit (engl.\ responsive), widerstandfähig (engl.\ resilient), elastisch (engl.\ elastic) und nachrichtenorientiert (engl.\ message driven) ist.~\citetext{\cite[S. 5]{vernon_reactive_2016}, \cite{boner_reactive_2014}}

Die Autoren des Reactive Manifestos glauben, dass die heutigen Anforderungen an moderne Software nur erfüllt werden können, wenn diese den vier Eigenschaften entspricht. Diese bezeichenen wir deshalb als reaktive Anwendungen.~\cite{boner_reactive_2014}\\
In dieser Arbeit möchte ich die Frage beantworten, welche Konzepte und Design Patterns helfen reaktive Systeme zu entwicklen.
