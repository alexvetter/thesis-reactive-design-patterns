\section{Design Patterns}
Viele der angesprochenen Probleme oder Lösungsansätze sind nicht neu. Programme wartbar und erweiterbar zu gestalten ist ebenfalls keine neue Anforderung. Beispielsweise ist der Ansatz, Komponenten voneinander zu entkoppeln seit Langem ein Ziel bei der Entwicklung von Software.\\
Die meisten Probleme sind wiederkehrende Probleme, die bereits schon einmal gelöst wurden. Ein Design Pattern (dt. Entwurfsmuster) ist eine Vorlage zur Lösung eines wiederkehrenden Problems. Der Architekt Christopher Alexander prägte den Begriff Design Pattern durch sein Buch \enquote{A Pattern Language}, in dem er eine allgemeingültige Sprache für wiederkehrende Probleme der Architektur definiert.\\
Zu Beginn des Buches beschreibt er die Sprache wie folgt~\cite[S. \textit{x}]{alexander_pattern_1977}:

\begin{quotation}
The elements of this language are entities called patterns. Each pattern describes a problem which occurs over and over again in our environment, and then describes the core of the solution to that problem, in such a way that you can use this solution a million times over, without ever doing it the same way twice.
\end{quotation}

Diese Definition haben Gamma et al. in ihrem Buch \enquote{Design Patterns} aufgegriffen und auf die Entwicklung objektorientierter Software übertragen. Ein Design Pattern bietet keine fertige Lösung sondern einen Lösungsansatz für generalisierte Probleme. Die Abstraktion von Problemen durch Design Patterns kann auf unterschiedlichen Ebenen erfolgen. Design Patterns können die Struktur bzw. Architektur von Systemen, deren Komponenten oder auch im Detail die Klassen beeinflussen~\cite[S.~3]{gamma_design_1995}~\cite[S.~127]{douglass_real-time_2003}. Deshalb können Patterns klassifiziert und gruppiert werden. Es gibt beispielsweise Architektur- oder Verhaltensmuster.\\
Ein Pattern beschreibt, ein generalisiertes Problem sowie einen entsprechenden allgemeinen Lösungsansatz. Durch die Definition von Design Patterns wird eine gemeinsame und allgemeingültige Sprache für die Probleme der Software Entwicklung geschaffen.\\
Ein Pattern besteht grundsätzlich aus vier Teilen~\cite[S.~3]{gamma_design_1995}:

\begin{enumerate}
\item Einem sprechenden und eindeutigen \textbf{Namen}, welcher das Problem und die Lösung in nicht mehr als zwei oder drei Wörtern beschreibt.
\item Eine Beschreibung des \textbf{Problems}, welches das Pattern zu lösen vermag. Wichtig ist hierbei auch der Kontext bzw. die Sichtweise auf das Problem.
\item Die eigentliche \textbf{Lösung} des Problems in Form von Diagrammen und Beschreibung. Die Lösung sollte jedoch allgemeingültig und technologieunabhängig sein.
\item Zudem sollten die \textbf{Konsequenzen} --- also Vor- und Nachteile, die durch die Verwendung des Pattern entstehen, genannt werden.
\end{enumerate}

Reaktives Software Design schließt von vornherein bereits einige Patterns aus --- vor allem bedingt durch die lose Kopplung sowie die asynchrone und nachrichtenorientierte Kommunikation zwischen den Komponenten. Als Folge dessen entstehen jedoch neue Problemfelder. Die Vergangenheit hat gezeigt, wie hilfreich Design Patterns bei der Lösung von Problemen sein können. Deshalb ist es nur sinnvoll sich mit Design Patterns zu beschäftigen, die ein reaktives Software Design im Sinne des Reactive Manifestos unterstützen~\cite[S.~54]{kuhn_reactive_2015}.
