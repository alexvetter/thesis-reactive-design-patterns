\section{Reactive Design Patterns}

Dieser Teil der Arbeit beschäftigt sich mit Reactive Design Patterns. Dabei handelt es sich nicht um gänzlich neue Design Patterns, sondern vielmehr, um Patterns, die sich für die Entwicklung reaktiver Systeme als nützlich erweisen. Der Fokus soll vorallem auf Patterns liegen, die die Eigenschaften reaktiver Anwendungen unterstützen.

\subsection{Observable Pattern}
Das reaktive Observable Pattern stammt aus dem Hause Microsoft und wurde von Erik Meijer in dem Framework für reaktive Programming \textit{Reactive Extensions} definiert. Die reaktive Programmierung beschäftigt sich mit dem Datenfluss einer Anwendung und mit der asynchronen Verarbeitung von Events. Bei der Entwicklung von Benutzeroberflächen ist reaktive Programmierung seit langem üblich und verbreitet. Eine Benutzeroberfläche muss stetig auf Eingaben, also Events, durch den Benutzer reagieren. Dazu gehören neben Tastatureingaben und Mausklicks auch Cursorbewegungen. Ebenso kann man aber auch Anfragen an eine Webapplikation als stetigen Datenstrom aus Events bezeichen.\\
Zu Beginn der Arbeit wurde bereits die Definition reaktiver Systeme genannt. Allgemeiner betrachtet bedeutet das englische Wort \textit{reactive} laut dem Merriam-Webster Wörterbuch \enquote{readily responsive to a stimulus}\footnote{http://www.merriam-webster.com/dictionary/reactive}. Demnach ist etwas \textit{reactive}, wenn es bereitwillig auf einen Reiz reagiert. Im Bezug auf Software sind Reize Events, die von einer Komponente verarbeitet werden müssen \cite{rappl_introduction_2016} \cite[S.~4]{carkci_dataflow_2014} \cite[S.~5]{blackheath_functional_2015}.\\
Für die Verarbeitung von Events nutzt man traditionellerweise das Verhaltensmuster \textit{Observer}. Dieses Muster beschreibt zwei Akteure bzw. Rollen. Das beobachtbare Subjekt, emittiert bzw. veröffentlicht Events. Die Events können von beliebig vielen Empfängern verarbeitet werden. Diese Empfänger werden als Beobachter oder auch als Observer bezeichnet. Ein Observer hat die Möglichkeit sich für Zustandsänderungen des Subjekts an- und abzumelden \cite[S.~293]{gamma_design_1995}.\\
Ein endlicher Strom von Events ist im Grunde nichts anderes als eine Liste. Für die Verarbeitung einer Liste nutzt man traditionellerweise das Verhaltensmuster \textit{Iterator}. Dieses Muster ermöglicht den sequenziellen Zugriff auf Elemente einer aggregierten Datenstruktur. Der Iterator definiert hierfür eine Schnittstelle für den Zugriff und das Traversieren beispielsweise einer Liste \cite[S.~257]{gamma_design_1995}.\\
Die beiden Patterns unterscheiden sich jedoch im Hinblick auf den Zugriff der Daten. Beim Iterator Pattern wird der Iterator nach weiteren Daten gefragt. Man nennt dies Pull-Strategie. Im Gegensatz dazu nutzt man beim Observer Pattern die Push-Strategie. Der Observer wird vom Subjekt benachrichtigt, falls weitere Daten verarbeiten werden müssen. Das Observer Pattern hat im Vergleich zum Iterator Pattern zwei Nachteile. Das Pattern beschreibt keine Möglichkeit einer Fehlerbehandlung und auch die Mitteilung, dass keine weiteren Events mehr eintreffen werden, ist nicht vorgesehen.\\
Das Observable Pattern kombiniert beide Patterns. Zum einen verwendet es die Push-Strategie und das Beobachten des Observer Patterns. Zum anderen kann ein Observable für das Ende des Datenstroms sowie über Fehler informiert werden. Ein Subscriber eines Observables definiert eine Schnittstelle mit den drei folgenden Methoden \cite{reactivex_2014}:

\begin{enumerate}
\item \textit{onNext(Event)}\\
Wird aufgerufen, falls ein Event verarbeitet werden soll.
\item \textit{onError(Exception)}\\
Wird aufgerufen, falls bspw. beim Aggregieren ein Fehler aufgetreten ist.
\item \textit{onCompleted()}\\
Wird aufgerufen, falls das Ende des Datenstroms erreicht ist.
\end{enumerate}

Observables abstrahieren den Datenfluss und durch die Observer Eigenschaften können Implementierungsdetails, wie beispielsweise Concurrency versteckt werden. Zudem wird eine asynchrone und nicht blockierende Verarbeitung von Events ermöglicht \cite[S.~81]{kuhn_reactive_2015}.

\pagebreak

\begin{lstlisting}[caption={Zusammenknüpfen und filtern zweiter Listen mit RxJava},label={lst:rxjava}]
List<String> w = Arrays.asList("Anna", "Eva", "Andrea", "Christiane");
List<String> m = Arrays.asList("Martin", "Tom", "Leon", "Alexander");
Observable.concat(Observable.from(w), Observable.from(m))
 .filter(name -> name.startsWith("A"))
 .subscribe(new Subscriber<String>() {
   public void onCompleted() {
    System.out.println("Completed");
   }
   public void onError(Throwable throwable) {
    System.err.println("Error");
   }
   public void onNext(String name) {
    System.out.println("Name: "+name);
   }
 });
\end{lstlisting}

Das Beispiel zeigt die Verwendung der Reactive Extensions für Java\footnote{RxJava (https://github.com/ReactiveX/RxJava)}. Es werden zwei Listen mit weiblichen und männlichen Vornamen verbunden (Zeile~3). Der daraus resultierende Datenstrom von Vornamen wird nach Namen gefiltert, welche mit \enquote{A} beginnen (Zeile~4). Zum Schluss wird ein Subscriber implemententiert, welcher die Namen ausgibt.\\
Anstelle der zwei Listen ist es auch möglich Anfragen von einem Sockets abzuarbeiten. Die einzelnen Funktionen, wie der Filter oder auch der Subscriber können wiederverwendet und anderweitig kombiniert werden.\\

Observables eignen sich sehr gut, um Datenströme zu verarbeiten. Die Verarbeitung kann asynchrone und nicht blockierend erfolgen. Die Reactive Extensions sind in ereignisbasierten Anwendungen (siehe \ref{subsec:eventdriven-concurrency}) von großem Nutzen. Für die Entwicklung von, durch Nachrichten gesteuerte, reaktive Applikationen sind die Reactive Extensions deshalb sehr nützlich. Jedoch wird durch deren Verwendung keine der vier reaktiven Eigenschaften direkt erfüllt \cite[S.~82]{kuhn_reactive_2015}.

\pagebreak

\subsection{Simple Component Pattern}\label{subsec:simple-component-pattern}
Ein umfangreiches System erledigt meist mehrere Aufgaben und hat verschiedenste Funktionalitäten. Die einzelnen Funktionen sollten isoliert voneinander betrachtet werden. Folglich sollte auch die Software in einzelne Komponenten aufgeteilt werden. Die Komplexität des großen Ganzen, kann auf viele kleinere Komponenten heruntergebrochen werden.\\
Dieses Pattern trägt den Namen Simple Component und wird wie folgt definiert:

\begin{quotation}
A component shall do only one thing, but do it in full \cite[S.~185]{kuhn_reactive_2015}.
\end{quotation}

Beispielsweise kann ein Textbearbeitungsprogramm mit Rechtschreibprüfung in Textbearbeitung und Rechtschreibprüfung aufgeteilt werden. Die Rechtschreibprüfung ist nicht abhängig von der eigentlichen Textbearbeitung und umgekehrt \cite[S.~185]{kuhn_reactive_2015}. Ein anderes Beispiel wäre ein Onlineshop. Ein Onlineshop besteht unter anderem aus folgenden Komponenten:

\begin{enumerate}
\item Produktverwaltung
\item Kundenverwaltung
\item Kundenauthentifizierung
\item Warenkorb \& Bestellvorgang
\item Zahlungsabwicklung
\end{enumerate}

Die Liste beansprucht keine Vollständigkeit und zeigt trotzdem, dass einzelne Komponenten keines Falls trival sein müssen aber im Vergleich zu der gesamten Anwendung simpler sind.

\pagebreak

Das Pattern ist in keinster Weise neu. Es kann aus dem \textit{Single Responsibility Priniciple} von Robert C. Martin abgeleitet werden. Das Prinzip zielt auf objektorientierte Systeme ab und lautet: \enquote{A class should have only one reason to change}. Befolgt man dieses Prinzip maximiert man die Kohäsion und minimiert die Kopplung zwischen Klassen --- oder auch zwischen Komponenten \cite[S.~185]{kuhn_reactive_2015} \cite{martin_single_2014}.\\
Eine weitere Definition in diesem Zusammenhang stammt von Rotem-Gal-Oz bezüglich eines Services in einer Service-orientierten Architektur:

\begin{quotation}
[...] a service should provide a distinct business function [...]. One of the characteristics of services is \textit{service autonomy}, which means the service should be mainly self-sufficient \cite[S.~7]{rotem_soa_2012}.
\end{quotation}

Ein Service oder im übertragenen Sinn eine Komponete sollte, falls möglich, autark und unabhängig sein. Folglich empfiehlt es sich die Logik einer Funktionalität nicht über mehrere Komponenten aufzuteilen.\\

Ein System im Ganzen betrachtet ist sehr komplex und schwer zu fassen. Mithilfe des Simple Component Patterns bricht man das System Schritt für Schritt in einzelne Verantwortlichkeiten und somit in einzele Komponenten auf. Es entsteht eine Hierarchie aus Komponenten und Unterkomponenten, die voneinander isoliert betrachtet werden können.\\
Das Simple Component Pattern ist ein grundlegendes und elementares Reactive Design Pattern. Folgt man dem Pattern können die reaktive Prinzipien \textit{supervision} (\ref{subsec:actor-model}) und \textit{share nothing} (\ref{subsec:sharenothing}) umgesetzt werden. Schlussendlich ermöglicht dieses Pattern, Komponenten von einander zu isolieren und dies wiederum unterstützt bei der Umsetzung der geforderten reaktiven Eigenschaft resilience (\ref{subsec:resilient}).

\pagebreak

\subsection{Let-It-Crash Pattern}
Prefer a full component restart to a complex internal failure handling. Vielleicht hier auch das Fail Fast Principle.

\pagebreak

\subsection{Heartbeat Pattern}

\pagebreak

\subsection{Circuit Breaker Pattern}

\pagebreak

\subsection{Bulkheads Pattern}

\pagebreak

\subsection{Idempotent Receiver}

\pagebreak

\subsection{Saga Pattern}
