% !TEX root = ../Masterarbeit.tex
\chapter{Schluss}
\vspace{-0.5cm}

\section{Reaktive Systeme}
Wie alle Programmierparadigmen oder Architekturansätze ist deren Nutzen abhängig vom Anwendungsfall. Dies gilt ebenfalls für reaktives Software Design auf Basis des Reactive Manifestos. In diesem Abschnitt werden abschließend nun drei Einsatzbereiche für reaktive Software vorgestellt.\\

\vspace{-0.5cm}
\subsection{Webapplikationen}
Reaktive Systeme eignen sich unteranderem für Webanwendungen mit beträchtlicher Interaktion und gewissen Echtzeitanforderungen. Diese müssen mit variabler Last und stetigen Anfragen bzw. Events umgehen können.\\
Immer mehr Unternehmen entwickeln Software für den Browser und das Internet. Denn die potenzielle Anzahl von Kunden durch das Internet mit seinen ca. 3 Milliarden Nutzer ist aus wirtschaftlicher Sicht gesehen sehr reizvoll. Auch die Verfügbarkeit auf jeglicher Art von Endgerät --- von Smartphone über Tablet zu Fernseher --- ist attraktiv.\\
Eine moderne Webanwendung besteht meist vollständig aus nutzerabhängigen und dynamischen Daten. Der Anteil der Nutzerinteraktion mit einer Webanwendung ist sehr hoch. Jede Interaktion ist einem Event gleichzusetzen und muss entsprechend verarbeitet werden. Auch möchten die Nutzer beispielsweise über Änderungen durch Andere so schnell wie möglich informiert werden --- nicht erst wenn die Seite neugeladen wurde. Die nachrichtenbasierte Kommunikation von reaktiver Software kann hier ideal eingesetzt werden.\\

\subsection{Microservices}

Bei dem Begriff Microservices handelt es sich um ein Architekturmuster, welches in den letzten Jahren populär wurde.\\
Komplexe Applikationen werden bei diesem Ansatz nicht mehr als eine monolitische Anwendung umgesetzt. Die Business Logik wird in einzelne Verantwortlichkeiten aufgeteilt. Im Vergleich zur traditionellen Modularisierung ist jeder fachliche Teil ein autonomer Prozess, genannt Microservice. Jeder Service ist für seine fachlichen Vorgänge und Daten verantwortlich und kommuniziert über Schnittstellen mit anderen Services. In Folge dessen kann jeder Service eigenständig entwickelt, versioniert und ausgeliefert werden.\\
Der Architekturansatz geht meisten einher mit organisatorischen Maßnahmen. Die Teams werde nicht mehr nach technischen Gesichtpunkten gebildet sondern nach fachlichen Verantwortlichkeiten. Entsprechend wird ein Microservice vollständig von einem Team entwickelt.\\

Diese Idee der Microservices lässt sich mithilfe des Reactive Manifestos und den Reactive Design Patterns optimal umsetzen, denn reaktive Systeme sind im Ansatz ähnlich. Aufgrund der Isolation der Komponenten durch das Simple Component Pattern ähneln diese Microservices. Jedoch wird bei Microservices das Prinzip der Supervision nicht explizt umgesetzt. Überdies ist die Kommunikation bei reaktiven Anwendungen zwischen den Komponenten zwingendermaßen asynchron und nachrichtenbasiert. Bei der Microservicearchitektur wird dies nicht vorgeschrieben und oft tendiert man bei der Umsetzung zum synchronen Protokoll HTTP --- mit den angesprochenen Nachteilen.\\
Die Microservicearchitektur hat zudem die reaktiven Eigenschaften \textit{resilience} und \textit{elasticity} nicht im Fokus. Jedoch können organisatorische Voraussetzung, wie sie bei Microservices im Grunde nötigt sind, auch bei der Umsetzung von reaktiven Systemen helfen.

% TODO Domain Driven Design
% TODO Transactional & Business Boundaries

% Die Geschäftsprozesse bzw. die fachlichen Vorgänge von Unternehmen bestehen in der Regel aus vielen einzelnen Schritten. Jeder Schritt kann einer Verantwortlichkeit zugeordnet werden. Die Vorgänge werden Schritt für Schritt abgearbeitet. Es bietet sich deshalb an Prozesse und deren Verantwortlichkeiten in einzelne Komponenten aufzuteilen und einen Vorgang als Nachrichtfluss abzubilden. Bildlich gesprochen werden die Vorgänge als Fließbandfertigung in Software umgesetzt. Die nachrichtenbasierte Kommunikation von reaktiver Software kann hier ideal eingesetzt werden.

\subsection{Internet of Things}
Das \enquote{Internet of Things} beschreibt den Wandel hinzu einer vernetzten Welt in der auch alltägliche Gegenstände miteinander verbunden sind und kommunizieren. Durch die immer kleiner werdenden eingebetteten Computer wird es möglich Geräte zu \enquote{intelligenten Gegenständen} zu machen.\\
Das Ziel wird es sein die physikalische und reale Welt mit dem Internet zu verbinden. Über Sensoren können dann automatisiert oder auf Kopfdruck Zustände wie beispielsweise die Raumtemperatur abgefragt werden. Mit diesen Informationen können dann andere intelligente Geräte über Aktoren beispielsweise die Heizung steuern.\\
Steckdosen, Lampenfassungen, Garagentore, Heizungen oder auch der Kühlschrank können über Sensoren und Aktoren erweitert und gesteuert werden.\\

Für diese Vielzahl von Clients, die miteinander kommunizieren müssen, ist eine nachrichtenbasierte Architektur die nicht-blockierende Kommunikation ermöglicht erstrebenswert. Das Internet of Things ist auch ein verteiltes Netzwerk und folglich entstehen ähnliche Probleme wie bei verteilten Anwendungen. Zur Koordination der intelligenten Geräte eignet sich deshalb ein System mit den Eigenschaften des Reactive Manifestos. 


\pagebreak

\section{Zusammenfassung}
Die Motivation hinter der Arbeit und dem Reactive Manifesto ist der Wandel der Anforderungen. Software muss heutzutage mit immer größeren Datenmengen und mit Millionen von Anfragen zurecht kommen. Der Wandel zeigt auch, dass die Datenmengen und die Anzahl der Anfragen rasant ansteigen --- deutlich rasanter als vor einigen Jahren. In Zukunft spielt auch die Machine-to-Machine Kommunikation immer mehr eine Rolle. Mit dem Internet of Things wird die Anzahl der \enquote{Nutzer} von Software erheblich steigen. Folglich muss Software gut skalieren sowohl auf einem modernen Multicore-Prozessor als auch in einem Computercluster.\\
Das Reactive Manifesto beschreibt vier Eigenschaften, die eine Anwendung erfüllen muss, um den Anforderungen von morgen gerecht zu werden. Anhand der Eigenschaften \textit{responsive}, \textit{elastic}, \textit{resilient} sowie \textit{message-driven} lässt sich ableiten, wie ein reaktives System aufgebaut sein muss.\\
Grundlegend für ein reaktives System ist ein Concurrency Modell, welches dem Entwickler die Möglichkeit bietet die Anwendung nebenläufig zu entwerfen. Dazu eignet sich vorrangig die ereignisbasierte Concurrency und auch die Actor-basierte Concurrency.\\
Design Patterns geben wiederkehrenden Problemen einen Namen und helfen dabei eine gemeinsame Sprache für die Probleme der Software Entwicklung zu finden. Ein Design Pattern bietet einen generalisierten Lösungsansätz für häufige und wiederkehrende Aufgaben. Die in dieser Arbeit beschriebenen Reactive Design Patterns helfen dabei reaktive Systeme zu entwerfen und umzusetzen. Alle ausgewählten Patterns unterstützen den Entwicklungsprozess einer reaktiven Anwendung im Sinne des Reactive Manifestos.\\
Die beschreibenen Reactive Design Patterns lassen sich in die fünf Gruppen Isolation, Loose Coupling, Failure Management, Concurrency und Latency Control aufteilen. Diese Gruppen spiegeln auch die Schwerpunkte der reaktiven Eigenschaften wieder. Beispielsweise ist Isolation und Loose Coupling sowohl für \textit{resilience} als auch für \textit{elasticity} wichtig.\\

Wie bereits erwähnt sollte --- vor der Umsetzung eines System nach den reaktiven Prinzipien --- geklärt werden, ob sich der Anwendungsfall dafür eignet. Dazu müssen die fachlichen Vorgänge analysiert werden und auch die Anforderungen als solche ausgewertet werden.\\
Die dogmatische Umsetzung des Reactive Manifestos ist nicht zwingend zielführend. Allerdings helfen die Reactive Design Patterns dabei Fehler, die andere bereits gemacht haben nicht wieder zu machen. Die Patterns weisen auch auf Umstände hin, mit denen man sich zuvor eventuell noch nicht beschäftigt hatte.\\
Manche Patterns wie Failfast oder Circuit Breaker lassen sich nachträglich in einem bestehenden System gut umsetzen um die \textit{resilience} zu verbessern. Andere Patterns oder Prinzipien hingegen sind im Nachhinein schwer zu bewerkstelligen, wie beispielsweise das Simple Component Pattern oder das Supervision Prinzip.
Das Auseinandersetzen und Aneignen egal welcher Design Patterns, ob reactive oder nicht, ist auf jeden Fall förderlich für die Qualität von Software jeglicher Art.