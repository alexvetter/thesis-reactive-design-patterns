% !TEX root = ../Masterarbeit.tex
\chapter{Schluss}

\section{Reaktive Systeme}
% TODO Domain Driven Design
% TODO Transactional & Business Boundaries
\subsection{Microservices}
\subsection{Internet of Things}

\section{Zusammenfassung}
Die Motivation hinter dem Reactive Manifesto ist der Wandel der Anforderungen. Software muss heutzutage mit immer größeren Datenmengen und mit Millionen von Anfragen zurecht kommen. Der Wandel zeigt auch, dass die Datenmengen und die Anzahl der Anfragen rasant ansteigen --- deutlich rasanter als vor einigen Jahren. In Zukunft spielt auch die Machine-to-Machine Kommunikation immer mehr eine Rolle. Mit dem Internet of Things wird die Anzahl der \enquote{Nutzer} einer Software erheblich steigen. Folglich muss Software gut skalieren sowohl auf einem modernen Multicore-Prozessor als auch in einem Computer Cluster. Das Reactive Manifesto beschreibt vier Eigenschaften, die eine Anwendung erfüllen muss um den Anforderungen von morgen gerecht zu werden. Anhand der Eigenschaften \textit{responsive}, \textit{elastic}, \textit{resilient} sowie \textit{message-driven} lässt sich ableiten, wie ein reaktives System aufgebaut sein muss.\\
Design Patterns geben wiederkehrenden Problemen einen Namen. Ein Design Pattern bietet einen generalisierten Lösungsansätz für häufige und wiederkehrende Aufgaben. Die in dieser Arbeit beschriebenen Reactive Design Patterns helfen dabei reaktive Systeme zu entwerfen und umzusetzen. Alle ausgewählten Patterns unterstützen den Entwicklungsprozess einer reaktiven Anwendung im Sinne des Reactive Manifestos.\\
